%
% Copyright (c) 2017 Lucio Andrés Illanes Albornoz <lucio@lucioillanes.de>
%
% {{{ Licence body
% Permission is hereby granted, free of charge, to any person obtaining a copy
% of this software and associated documentation files (the "Software"), to deal
% in the Software without restriction, including without limitation the rights
% to use, copy, modify, merge, publish, distribute, sublicense, and/or sell
% copies of the Software, and to permit persons to whom the Software is
% furnished to do so, subject to the following conditions:
%
% The above copyright notice and this permission notice shall be included in all
% copies or substantial portions of the Software.
%
% THE SOFTWARE IS PROVIDED "AS IS", WITHOUT WARRANTY OF ANY KIND, EXPRESS OR
% IMPLIED, INCLUDING BUT NOT LIMITED TO THE WARRANTIES OF MERCHANTABILITY,
% FITNESS FOR A PARTICULAR PURPOSE AND NONINFRINGEMENT. IN NO EVENT SHALL THE
% AUTHORS OR COPYRIGHT HOLDERS BE LIABLE FOR ANY CLAIM, DAMAGES OR OTHER
% LIABILITY, WHETHER IN AN ACTION OF CONTRACT, TORT OR OTHERWISE, ARISING FROM,
% OUT OF OR IN CONNECTION WITH THE SOFTWARE OR THE USE OR OTHER DEALINGS IN THE
% SOFTWARE.
% }}}
%

\documentclass{article}

% {{{ Packages
\usepackage{caption}
\usepackage{enumitem}
\usepackage{fancyhdr}
\usepackage{geometry}
\usepackage{iflang}
\usepackage{tabularx}
\usepackage{url}
\usepackage{xifthen}
\usepackage[final]{pdfpages}
\usepackage[ngerman,british]{babel}
\usepackage[T1]{fontenc}
\usepackage[utf8]{inputenc}
% }}}
% {{{ Formats and types
\captionsetup[table]{
  aboveskip=7.5pt,
  font={bf,small},
  labelformat=empty,
  singlelinecheck=false}
\newcolumntype{b}{
  p{150pt}|}
\newcolumntype{s}{
  *5{>{\arraybackslash}X}}
\setenumerate{
  itemsep=0pt,
  parsep=0pt}
% }}}
% {{{ Page setup
\geometry{
  a4paper,
  margin=0.5in,
  top=0.75in}
\makeatletter
\setlength{\@fptop}{0pt}
\makeatother
\fancyhf{}
\pagestyle{fancy}
% }}}

\begin{document}
% {{{ Internationalisation
\babeltags{de = ngerman}
\babeltags{en = british}
\expandafter\selectlanguage\expandafter{\varLang}
% }}}
% {{{ Header legend
\ifthenelse{\equal{\varPrivate}{1}} {
  \IfLanguageName{british}{
    \chead{Last update: 27th May 2017 · Character references included on pages 2-3}
  } {
    \chead{Zuletzt geändert am: 27. Mai 2017 · Enthält Referenzen auf Seiten 2-3}
  }
} {
  \IfLanguageName{british}{
    \chead{Last update: 27th May 2017 · Public version, character references available on request}
  } {
    \chead{Zuletzt geändert am: 27. Mai 2017 · Öffentliche Version, Referenzen auf Anfrage verfügbar}
  }
}
% }}}

\begin{table}
  \centering{\title*{\huge{Curriculum Vitae}}}
  % {{{ Personal details
  \IfLanguageName{british}{
    % {{{ English
    \caption{}
    \begin{tabularx}{\linewidth}{bs}
      \bf{Name} & Lucio Andrés Illanes Albornoz\\
      Email address & \url{lucio@lucioillanes.de}\\
      Mobile telephone number & +49 1623 714 810\\
      Address & Nagelsweg 16, 20097 Hamburg, Germany\\
      Date of Birth & 13 August 1987\\
      Nationalities & Chilean, German\\
      GitHub & \url{https://git.io/vS4CG}\\
      Website & \url{https://www.lucioillanes.de/}
    \end{tabularx}
    % }}}
  }{
    % {{{ German
    \caption{}
    \begin{tabularx}{\linewidth}{bs}
      \bf{Name} & Lucio Andrés Illanes Albornoz\\
      E-Mail-Adresse & \url{lucio@lucioillanes.de}\\
      Telefonnummer & +49 1623 714 810\\
      Adresse & Nagelsweg 16, 20097 Hamburg, Deutschland\\
      Geburtstag & 13. August 1987\\
      Staatsangehörigkeit & Chilenisch, Deutsch\\
      GitHub & \url{https://git.io/vS4CG}\\
      Webseite & \url{https://www.lucioillanes.de/}
    \end{tabularx}
    % }}}
  }
  % }}}
  % {{{ Software development
  \IfLanguageName{british}{
    % {{{ English
    \caption{Software development}
    \begin{tabularx}{\linewidth}{bs}
      Programming areas & Systems \& network programming on Linux, *BSD \& Windows\\
      Languages and tools & C, Tcl, BSD \& GNU development environments, gdb, Qemu; amd/x86 \& MIPS assembler, C++, Lua, Python \& WinDbg: some experience
    \end{tabularx}
    % }}}
  }{
    % {{{ German
    \caption{Softwareentwicklung}
    \begin{tabularx}{\linewidth}{bs}
      Hauptfelder & System- \& Netzwerkprogramming auf Linux, *BSD \& Windows\\
      Sprachen und Tools & C, Tcl, BSD- \& GNU-Entwicklungsumgebungen, gdb, Qemu; ferner amd/x86- \& MIPS-Assembler, C++, Lua, Python \& WinDbg
    \end{tabularx}
    % }}}
  }
  % }}}
  % {{{ System administration
  \IfLanguageName{british}{
    % {{{ English
    \caption{System administration}
    \begin{tabularx}{\linewidth}{bs}
      Platforms and servers & Debian/-derived Linux, Free/Net/OpenBSD, ircd-hybrid, nginx \\
      Scripting and automation & Ansible, Bourne/Bash/Z shell, BSD rc, systemd, Perl, rsync\\
      Roles and technologies & iptables \& pf-based firewalls, IPsec \& OpenVPN, LXC \& cgroups, ZFS
    \end{tabularx}
    % }}}
  }{
    % {{{ German
    \caption{Systemsadministration}
    \begin{tabularx}{\linewidth}{bs}
      Systeme und Server & Debian/-deriviertes Linux, Free/Net/OpenBSD, ircd-hybrid, nginx\\
      Scripting und Automatisierung & Ansible, Bourne/Bash/Z shell, BSD rc, systemd, Perl, rsync\\
      Serverrollen und Technologien & iptables \& pf-basierte Firewalls, IPsec \& OpenVPN, LXC \& cgroups, ZFS
    \end{tabularx}
    % }}}
  }
  % }}}
  % {{{ Project involvement
  \IfLanguageName{british}{
    % {{{ English
    \caption{Project involvement}
    \begin{tabularx}{\linewidth}{bs}
      February 2016-ongoing & \bf{midipix\_build: build/cross-compilation infrastructure for midipix, a POSIX/Linux-compatible development/runtime environment for Windows}
    \end{tabularx}
    \begin{enumerate}
      \item Automatically cross-builds ca. 150 packages covering the toolchain, runtime components \& 3rd party software.
      \item Written in reasonably portable Bourne shell, ca. 1500 SLOC, pragmatically modular.
      \item Flexible \& lightweight infrastructure - has proven itself towards maturity over the course of ca. 1 year.
    \end{enumerate}
    \begin{tabularx}{\linewidth}{bs}
      November 2016-December 2016 & \bf{Tcl TIP \#458: design and implementation of epoll/kqueue support in the Tcl notifier on Linux/*BSD, resp. for Tcl, a high-level, general-purpose, interpreted, dynamic programming language (FlightAware bounty programme)}
    \end{tabularx}
    \begin{enumerate}
      \item Tcl's original event-based architecture based on select(2) had scalability and code complexity concerns.
      \item On Linux/*BSD, epoll/kqueue is now used instead, which eliminates a major bottleneck in event-heavy code.
      \item Code complexity was reduced as the code is no longer unnecessarily multi-threaded, without breaking multi-threaded Tcl or C code.
    \end{enumerate}
    % }}}
  }{
    % {{{ German
    \caption{Projektbeteiligungen}
    \begin{tabularx}{\linewidth}{bs}
      Seit Februar 2016 & \bf{midipix\_build: build/cross-compilation-Infrastruktur für Midipix, eine POSIX/Linux-kompatible Entwicklungs/Endnutzerumgebung für Windows}
    \end{tabularx}
    \begin{enumerate}
      \item Automatisierter Cross-build von ca. 150 Paketen wie der Toolchain, Runtime components \& Software dritter.
      \item Umgesetzt als relativ portables Bourne-Skellskript, ca. 1500 SLOC, pragmatisch Modular.
      \item Flexible \& lightweight Infrastruktur - hat sich über den Verlauf von ca. 1 Jahr als Zuverlässig bewährt.
    \end{enumerate}
    \begin{tabularx}{\linewidth}{bs}
      November 2016-Dezember 2016 & \bf{Tcl TIP \#458: Entwurf und Umsetzung der epoll/kqueue-Unterstützung im Tcl notifier auf Linux bzw. *BSD (FlightAware bounty programme)}
    \end{tabularx}
    \begin{enumerate}
      \item Tcl's ursprüngliche Event-basierte Architektur auf der Basis von select(2) hat Skalabilitäts- und Codekomplexitätsprobleme.
      \item Auf Linux/*BSD wird nun epoll/kqueue benutzt, wodurch ein bedeutender Flaschenhals in I/O-gebundenem Code eliminiert wurde.
      \item Codekomplexität wurde verringert, da der neue Code lediglich einen Thread benutzt, ohne dabei Tcl oder C code mit Multi-threading nicht mehr zu unterstützen.
    \end{enumerate}
    % }}}
  }
  % }}}
  % {{{ Work experience
  \IfLanguageName{british}{
    % {{{ English
    \caption{Work experience}
    \begin{tabularx}{\linewidth}{bs}
      2003 & \bf{Sasol Germany, Hamburg} (Internship)\\
      Responsibilities/role & Internal help desk/support on Windows platforms\\
      2002 & \bf{Compaq Computer Corporation, Hamburg} (Internship)\\
      Responsibilities/role & Corporate customer support on OpenVMS platforms
    \end{tabularx}
    % }}}
  }{
    % {{{ German
    \caption{Praktika}
    \begin{tabularx}{\linewidth}{bs}
      2003 & \bf{Sasol Germany, Hamburg}\\
      Verantwortungen/Aufgaben & Betriebsinternes Help desk/Support auf Windows-Systemen\\
      2002 & \bf{Compaq Computer Corporation, Hamburg}\\
      Verantwortungen/Aufgaben & Firmenkundenbetreuung auf OpenVMS-Systemen
    \end{tabularx}
    % }}}
  }
  % }}}
  % {{{ Languages (CEFR levels given are personal judgments unless specified otherwise)
  \IfLanguageName{british}{
    % {{{ English
    \caption{Languages (CEFR levels given are personal judgments unless specified otherwise)}
    \begin{tabularx}{\linewidth}{bs}
      Paternal language & Spanish (CEFR: C1)\\
      Maternal language & German (CEFR: C1)\\
      1st foreign language & English (CEFR: C1; at Colón Language Center Hamburg: A)\\
      2nd foreign language & Classical Arabic, some dialects (CEFR: A2)
    \end{tabularx}
    % }}}
  }{
    % {{{ German
    \caption{Sprachen (Nach persönlicher Beurteilung sofern nicht anders angegeben)}
    \begin{tabularx}{\linewidth}{bs}
      Vatersprache & Spanisch (GeR: C1)\\
      Muttersprache & Deutsch (GeR: C1)\\
      1ste Fremdsprache & Englisch (GeR: C1; Colón Language Center Hamburg: 1)\\
      2te Fremdsprache & Klassisches Arabisch, manche Dialekte (GeR: A2)
    \end{tabularx}
    % }}}
  }
  % }}}
  % {{{ Educational career
  \IfLanguageName{british}{
    % {{{ English
    \caption{Educational career}
    \begin{tabularx}{\linewidth}{bs}
      Degree(s) attained & \bf{Mittlere Reife/MSA, equivalent to GCSE in the UK}\\
      Sep. 2015-Jan. 2017 & Deutsche Angestelltenakademie (MSA course)\\
      Aug. 2001-Feb. 2004 & Integrierte Gesamtschule Walddörfer, Hamburg
    \end{tabularx}
    % }}}
  }{
    % {{{ German
    \caption{Bildung}
    \begin{tabularx}{\linewidth}{bs}
      Schulabschluss & \bf{Mittlere Reife}\\
      Sep. 2015-Jan 2017 & Deutsche Angestelltenakademie (MSA-Kurs)\\
      Aug. 2001-Feb. 2004 & Integrierte Gesamtschule Walddörfer, Hamburg
    \end{tabularx}
    % }}}
  }
  % }}}
\end{table}

% {{{ Character references
\ifthenelse{\equal{\varPrivate}{1}} {
  \clearpage
  \includepdf[pages=-, offset=5 -5]{Reference1.pdf}
}
% }}}

\end{document}

% vim:expandtab fenc=utf-8 sw=2 ts=2 tw=0
